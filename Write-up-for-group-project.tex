% Options for packages loaded elsewhere
\PassOptionsToPackage{unicode}{hyperref}
\PassOptionsToPackage{hyphens}{url}
%
\documentclass[
]{article}
\usepackage{amsmath,amssymb}
\usepackage{iftex}
\ifPDFTeX
  \usepackage[T1]{fontenc}
  \usepackage[utf8]{inputenc}
  \usepackage{textcomp} % provide euro and other symbols
\else % if luatex or xetex
  \usepackage{unicode-math} % this also loads fontspec
  \defaultfontfeatures{Scale=MatchLowercase}
  \defaultfontfeatures[\rmfamily]{Ligatures=TeX,Scale=1}
\fi
\usepackage{lmodern}
\ifPDFTeX\else
  % xetex/luatex font selection
\fi
% Use upquote if available, for straight quotes in verbatim environments
\IfFileExists{upquote.sty}{\usepackage{upquote}}{}
\IfFileExists{microtype.sty}{% use microtype if available
  \usepackage[]{microtype}
  \UseMicrotypeSet[protrusion]{basicmath} % disable protrusion for tt fonts
}{}
\makeatletter
\@ifundefined{KOMAClassName}{% if non-KOMA class
  \IfFileExists{parskip.sty}{%
    \usepackage{parskip}
  }{% else
    \setlength{\parindent}{0pt}
    \setlength{\parskip}{6pt plus 2pt minus 1pt}}
}{% if KOMA class
  \KOMAoptions{parskip=half}}
\makeatother
\usepackage{xcolor}
\usepackage[margin=1in]{geometry}
\usepackage{color}
\usepackage{fancyvrb}
\newcommand{\VerbBar}{|}
\newcommand{\VERB}{\Verb[commandchars=\\\{\}]}
\DefineVerbatimEnvironment{Highlighting}{Verbatim}{commandchars=\\\{\}}
% Add ',fontsize=\small' for more characters per line
\usepackage{framed}
\definecolor{shadecolor}{RGB}{248,248,248}
\newenvironment{Shaded}{\begin{snugshade}}{\end{snugshade}}
\newcommand{\AlertTok}[1]{\textcolor[rgb]{0.94,0.16,0.16}{#1}}
\newcommand{\AnnotationTok}[1]{\textcolor[rgb]{0.56,0.35,0.01}{\textbf{\textit{#1}}}}
\newcommand{\AttributeTok}[1]{\textcolor[rgb]{0.13,0.29,0.53}{#1}}
\newcommand{\BaseNTok}[1]{\textcolor[rgb]{0.00,0.00,0.81}{#1}}
\newcommand{\BuiltInTok}[1]{#1}
\newcommand{\CharTok}[1]{\textcolor[rgb]{0.31,0.60,0.02}{#1}}
\newcommand{\CommentTok}[1]{\textcolor[rgb]{0.56,0.35,0.01}{\textit{#1}}}
\newcommand{\CommentVarTok}[1]{\textcolor[rgb]{0.56,0.35,0.01}{\textbf{\textit{#1}}}}
\newcommand{\ConstantTok}[1]{\textcolor[rgb]{0.56,0.35,0.01}{#1}}
\newcommand{\ControlFlowTok}[1]{\textcolor[rgb]{0.13,0.29,0.53}{\textbf{#1}}}
\newcommand{\DataTypeTok}[1]{\textcolor[rgb]{0.13,0.29,0.53}{#1}}
\newcommand{\DecValTok}[1]{\textcolor[rgb]{0.00,0.00,0.81}{#1}}
\newcommand{\DocumentationTok}[1]{\textcolor[rgb]{0.56,0.35,0.01}{\textbf{\textit{#1}}}}
\newcommand{\ErrorTok}[1]{\textcolor[rgb]{0.64,0.00,0.00}{\textbf{#1}}}
\newcommand{\ExtensionTok}[1]{#1}
\newcommand{\FloatTok}[1]{\textcolor[rgb]{0.00,0.00,0.81}{#1}}
\newcommand{\FunctionTok}[1]{\textcolor[rgb]{0.13,0.29,0.53}{\textbf{#1}}}
\newcommand{\ImportTok}[1]{#1}
\newcommand{\InformationTok}[1]{\textcolor[rgb]{0.56,0.35,0.01}{\textbf{\textit{#1}}}}
\newcommand{\KeywordTok}[1]{\textcolor[rgb]{0.13,0.29,0.53}{\textbf{#1}}}
\newcommand{\NormalTok}[1]{#1}
\newcommand{\OperatorTok}[1]{\textcolor[rgb]{0.81,0.36,0.00}{\textbf{#1}}}
\newcommand{\OtherTok}[1]{\textcolor[rgb]{0.56,0.35,0.01}{#1}}
\newcommand{\PreprocessorTok}[1]{\textcolor[rgb]{0.56,0.35,0.01}{\textit{#1}}}
\newcommand{\RegionMarkerTok}[1]{#1}
\newcommand{\SpecialCharTok}[1]{\textcolor[rgb]{0.81,0.36,0.00}{\textbf{#1}}}
\newcommand{\SpecialStringTok}[1]{\textcolor[rgb]{0.31,0.60,0.02}{#1}}
\newcommand{\StringTok}[1]{\textcolor[rgb]{0.31,0.60,0.02}{#1}}
\newcommand{\VariableTok}[1]{\textcolor[rgb]{0.00,0.00,0.00}{#1}}
\newcommand{\VerbatimStringTok}[1]{\textcolor[rgb]{0.31,0.60,0.02}{#1}}
\newcommand{\WarningTok}[1]{\textcolor[rgb]{0.56,0.35,0.01}{\textbf{\textit{#1}}}}
\usepackage{graphicx}
\makeatletter
\def\maxwidth{\ifdim\Gin@nat@width>\linewidth\linewidth\else\Gin@nat@width\fi}
\def\maxheight{\ifdim\Gin@nat@height>\textheight\textheight\else\Gin@nat@height\fi}
\makeatother
% Scale images if necessary, so that they will not overflow the page
% margins by default, and it is still possible to overwrite the defaults
% using explicit options in \includegraphics[width, height, ...]{}
\setkeys{Gin}{width=\maxwidth,height=\maxheight,keepaspectratio}
% Set default figure placement to htbp
\makeatletter
\def\fps@figure{htbp}
\makeatother
\setlength{\emergencystretch}{3em} % prevent overfull lines
\providecommand{\tightlist}{%
  \setlength{\itemsep}{0pt}\setlength{\parskip}{0pt}}
\setcounter{secnumdepth}{-\maxdimen} % remove section numbering
\ifLuaTeX
  \usepackage{selnolig}  % disable illegal ligatures
\fi
\IfFileExists{bookmark.sty}{\usepackage{bookmark}}{\usepackage{hyperref}}
\IfFileExists{xurl.sty}{\usepackage{xurl}}{} % add URL line breaks if available
\urlstyle{same}
\hypersetup{
  pdftitle={Group project - Final},
  hidelinks,
  pdfcreator={LaTeX via pandoc}}

\title{Group project - Final}
\author{}
\date{\vspace{-2.5em}2024-03-31}

\begin{document}
\maketitle

Intro Explain why we used BMI as scoring system

\hypertarget{introduction}{%
\section{Introduction}\label{introduction}}

This report investigates obesity and its related health risks using a
dataset on individual characteristics and habits. Our primary goal was
to identify significant factors influencing Body Mass Index (BMI) and
use this information to construct a scoring system for obesity risk and
overall health risk.

BMI is a commonly used metric to assess weight status and potential
obesity risk. It is calculated using an individual's weight and height.
While BMI is a valuable tool, it has limitations. For instance, it
doesn't account for muscle mass, which can be high in some individuals
with healthy weight.

This report acknowledges these limitations but utilizes BMI as a
starting point for our analysis. We explore the factors impacting BMI
and leverage this knowledge to build a multi-variable scoring system for
both obesity risk and health risk associated with high BMI.

\hypertarget{methods}{%
\section{Methods}\label{methods}}

After viewing the correlation Matrix and computing BMI as the Gold
standard, out first step in gathering information on our data was to
make a linear regression model in order to find the significant
predictors of BMI from the relevant variables.

Since BMI is calculated using weight/Height\^{}2 and Obesity level is
calculated based on BMI, we first needed to clean the data to build an
appropriate model by removing these variables.

\begin{verbatim}
## 
## Call:
## lm(formula = BMI ~ ., data = train)
## 
## Residuals:
##      Min       1Q   Median       3Q      Max 
## -17.4414  -3.9418   0.3361   3.4788  23.7055 
## 
## Coefficients:
##                                         Estimate Std. Error t value Pr(>|t|)
## (Intercept)                              3.35885    6.13000   0.548 0.583823
## GenderMale                              -0.82875    0.33596  -2.467 0.013753
## Age                                      0.31405    0.03353   9.368  < 2e-16
## Family_History_w_Overweightyes           6.80735    0.44395  15.334  < 2e-16
## HiCal_Food_Consumpyes                    2.32198    0.50496   4.598 4.65e-06
## Veggie_Consump                           3.00114    0.30679   9.782  < 2e-16
## Main_Meal_Consump                        0.47339    0.20460   2.314 0.020827
## Food_bw_MealsFrequently                 -4.03993    1.05179  -3.841 0.000128
## Food_bw_Mealsno                          1.88574    1.36437   1.382 0.167153
## Food_bw_MealsSometimes                   3.10074    0.97915   3.167 0.001575
## Does_Smokeyes                           -0.40957    1.05380  -0.389 0.697589
## Water_Consump                            0.57999    0.26596   2.181 0.029369
## Monitor_Caloriesyes                     -2.33645    0.74096  -3.153 0.001649
## Physical_Activ_Amt                      -0.68224    0.19496  -3.499 0.000481
## Tech_Time                               -0.63740    0.27248  -2.339 0.019463
## Alcohol_ConsumpFrequently               -3.36754    5.90954  -0.570 0.568873
## Alcohol_Consumpno                       -5.04959    5.84758  -0.864 0.387992
## Alcohol_ConsumpSometimes                -2.48026    5.85110  -0.424 0.671707
## Transportation_UseBike                  -0.84260    4.12896  -0.204 0.838328
## Transportation_UseMotorbike              5.78189    1.86771   3.096 0.002003
## Transportation_UsePublic_Transportation  5.39302    0.50164  10.751  < 2e-16
## Transportation_UseWalking                2.59810    1.12729   2.305 0.021329
##                                            
## (Intercept)                                
## GenderMale                              *  
## Age                                     ***
## Family_History_w_Overweightyes          ***
## HiCal_Food_Consumpyes                   ***
## Veggie_Consump                          ***
## Main_Meal_Consump                       *  
## Food_bw_MealsFrequently                 ***
## Food_bw_Mealsno                            
## Food_bw_MealsSometimes                  ** 
## Does_Smokeyes                              
## Water_Consump                           *  
## Monitor_Caloriesyes                     ** 
## Physical_Activ_Amt                      ***
## Tech_Time                               *  
## Alcohol_ConsumpFrequently                  
## Alcohol_Consumpno                          
## Alcohol_ConsumpSometimes                   
## Transportation_UseBike                     
## Transportation_UseMotorbike             ** 
## Transportation_UsePublic_Transportation ***
## Transportation_UseWalking               *  
## ---
## Signif. codes:  0 '***' 0.001 '**' 0.01 '*' 0.05 '.' 0.1 ' ' 1
## 
## Residual standard error: 5.705 on 1385 degrees of freedom
## Multiple R-squared:  0.5017, Adjusted R-squared:  0.4941 
## F-statistic: 66.39 on 21 and 1385 DF,  p-value: < 2.2e-16
\end{verbatim}

\begin{verbatim}
## [1] "RMSE:"
\end{verbatim}

\begin{verbatim}
## [1] 5.729089
\end{verbatim}

From the summary of our model, we noted that our R\^{}2 is .49 which
means only half the variability of BMI was captured by this data. This
is normal for a dataset of this nature which deals with predicting
humans. This low value is also justified because our data is missing
major predictors of obesity like calorie intake.

Based on the model we were able to find the most significant predictors
for BMI from the P-values provided

We were also able to find out how accurate these predictors were in
predicting an individuals level of obesity by performing multiple
logistic regression using Obesity level as the target.

\begin{Shaded}
\begin{Highlighting}[]
\NormalTok{accuracy}
\end{Highlighting}
\end{Shaded}

\begin{verbatim}
## [1] 0.4758523
\end{verbatim}

The accuracy of our predictors ability to predict obesity group was
around 45\% which is standard for data of this nature. It also means
that we need more information than the variables provided to fully
predict the Obesity group for an individual.

Based on results of first model (regression) we tested the
impact/dependency between the significant predictors BMI, and BMI using
Kruskal-Wallis tests. If a predictor had a significant impact on BMI, we
then performed a Chi-Square test to determine if this relationship was
significant or random using the Obesity level, a health based grouping
of BMI values. Based on these test results we were be able to determine
the significant predictors of an individuals BMI.

Also, in order to test the effects of BMI on an individuals health we
performed a Kruskal-Wallis test to measure the effect of BMI on an
individuals age as well as the effect of Obesity level on age.

As a result of the aforementioned tests, every variable that passed the
KW test also passed the Chi-Square test and as a result we were able to
establish the significant predictors of obesity:

Family History of Overweight

Hi Calorie Food Consumption

Physical Activity

Food between Meals

Water Consumption

Monitor Calories

Transportation Use

Tech Time

From this, we created several data visualizations to further prove the
significance of the mentioned predictors. Out of all the data
visualizations the variables with family history with overweight, high
calorie food consumption, physical activity and monitor calories gave us
the most useful information.

\includegraphics{Write-up-for-group-project_files/figure-latex/unnamed-chunk-6-1.pdf}

The obesity level vs family history with overweight bar plot (figure 1)
shows us the significant change in the number of people with an obesity
level of overweight and higher when they have a family history with
overweight.

\includegraphics{Write-up-for-group-project_files/figure-latex/unnamed-chunk-7-1.pdf}

The obesity level vs high calorie food consumption (figure 2) shows the
significant change in proportion of normal weight people and people
categorized under an obesity level.

\includegraphics{Write-up-for-group-project_files/figure-latex/unnamed-chunk-8-1.pdf}
The obesity level vs physical activity amount violin plot (figure 3)
shows us that there is a high density of people between normal weight
and obesity type 2 with physical activity amounts between 0.5 and 1.5
while obesity type 3 have a high density of people with physical
activity amounts between 0 and 0.5.

\includegraphics{Write-up-for-group-project_files/figure-latex/unnamed-chunk-9-1.pdf}
The obesity level vs calorie monitoring bar plot (figure 4) shows us
that there appears to be no significant difference between people who do
not monitor their calories, however, the proportion of people who
monitor their calories changes between obesity levels with the count of
people monitoring their calories start to decrease when they have an
obesity level of overweight 2 or higher.

We also learned that while BMI itself doesn't have a significant impact
on age, an individuals obesity level does.

Armed with these insights, we went on to determine how much obesity
level impacts age by performing a logistic regression predicting age
using BMI. Before we did this, we plotted and analyzed a regression line
of BMI and age at different age ranges to visualize this relationship.

\begin{verbatim}
## `geom_smooth()` using formula = 'y ~ x'
\end{verbatim}

\includegraphics{Write-up-for-group-project_files/figure-latex/unnamed-chunk-10-1.pdf}

\begin{verbatim}
## `geom_smooth()` using formula = 'y ~ x'
\end{verbatim}

\includegraphics{Write-up-for-group-project_files/figure-latex/unnamed-chunk-10-2.pdf}

\begin{verbatim}
## `geom_smooth()` using formula = 'y ~ x'
\end{verbatim}

\includegraphics{Write-up-for-group-project_files/figure-latex/unnamed-chunk-10-3.pdf}

\begin{verbatim}
## `geom_smooth()` using formula = 'y ~ x'
\end{verbatim}

\includegraphics{Write-up-for-group-project_files/figure-latex/unnamed-chunk-10-4.pdf}

Based on the plots, it was apparent that as the age of the observations
increase, the maximum BMI observed decreases. We then extrapolated that
younger individuals tend to have higher BMIs as physically, their bodies
are not stable or fully developed yet. We then inferred that as one
ages, a high BMI can lead to health complications which could be the
reason for a decrease in observations. Based on these result we then
performed our logistic regression.

In order to avoid skewing the data, as BMI can often be misinterpreted
for younger and developing individuals, we took a look at the effect of
type 2 obesity (BMI \textgreater{} 35) on whether an individual is older
than 40 years old.

\begin{Shaded}
\begin{Highlighting}[]
\NormalTok{data2}\OtherTok{\textless{}{-}}\NormalTok{data}
\NormalTok{data2}\SpecialCharTok{$}\NormalTok{Age40 }\OtherTok{\textless{}{-}} \FunctionTok{ifelse}\NormalTok{(data2}\SpecialCharTok{$}\NormalTok{Age }\SpecialCharTok{\textgreater{}} \DecValTok{40}\NormalTok{, }\DecValTok{1}\NormalTok{, }\DecValTok{0}\NormalTok{)}
\NormalTok{data2}\SpecialCharTok{$}\NormalTok{BMI35 }\OtherTok{\textless{}{-}} \FunctionTok{ifelse}\NormalTok{(data2}\SpecialCharTok{$}\NormalTok{BMI }\SpecialCharTok{\textgreater{}} \DecValTok{35}\NormalTok{, }\DecValTok{1}\NormalTok{, }\DecValTok{0}\NormalTok{)}
 \CommentTok{\# Fit logistic regression model}
\NormalTok{log\_model4035 }\OtherTok{\textless{}{-}} \FunctionTok{glm}\NormalTok{(Age40 }\SpecialCharTok{\textasciitilde{}}\NormalTok{ BMI35, }\AttributeTok{data =}\NormalTok{ data2, }\AttributeTok{family =}\NormalTok{ binomial)}
\FunctionTok{summary}\NormalTok{(log\_model4035)}
\end{Highlighting}
\end{Shaded}

\begin{verbatim}
## 
## Call:
## glm(formula = Age40 ~ BMI35, family = binomial, data = data2)
## 
## Coefficients:
##             Estimate Std. Error z value Pr(>|z|)    
## (Intercept)  -3.4795     0.1514 -22.990   <2e-16 ***
## BMI35        -0.4224     0.3285  -1.286    0.198    
## ---
## Signif. codes:  0 '***' 0.001 '**' 0.01 '*' 0.05 '.' 0.1 ' ' 1
## 
## (Dispersion parameter for binomial family taken to be 1)
## 
##     Null deviance: 524.20  on 2110  degrees of freedom
## Residual deviance: 522.42  on 2109  degrees of freedom
## AIC: 526.42
## 
## Number of Fisher Scoring iterations: 6
\end{verbatim}

Overall, based on these results, there was no statistically significant
evidence to suggest that having a BMI over 35 affects the likelihood of
an individual being younger than 40 years old, as the coefficient for
BMI35 was not significant. However we used these results to find the
likelihood that an individual is over 40 if their BMI is greater than 35

intercept (Estimated Coefficient: -3.4795): The intercept represents the
estimated log odds of an individual being younger than 40 years old when
they do not have a BMI over 35. A negative coefficient suggests that
individuals who do not have a BMI over 35 are less likely to be younger
than 40 years old.

BMI35 (Estimated Coefficient: -0.4224): The coefficient for BMI35
represents the change in the log odds of an individual being younger
than 40 years old when they have a BMI over 35 compared to when they do
not have a BMI over 35. Here, however, we're interested in how it
affects the likelihood of an individual being older than 40 years old.
Given that the coefficient is negative, it implies that individuals with
a BMI over 35 are less likely to be older than 40 years old.

Now from this we were able to find the probability an individual is over
40 years old if their BMI is greater than 35 (Obesity type 2)

\begin{Shaded}
\begin{Highlighting}[]
\NormalTok{intercept }\OtherTok{\textless{}{-}} \SpecialCharTok{{-}}\FloatTok{3.4795}
\NormalTok{BMI35\_coefficient }\OtherTok{\textless{}{-}} \SpecialCharTok{{-}}\FloatTok{0.4224}

\CommentTok{\# BMI value indicating over 35}
\NormalTok{BMI\_over\_35 }\OtherTok{\textless{}{-}} \DecValTok{1}

\CommentTok{\# Calculate log odds}
\NormalTok{log\_odds }\OtherTok{\textless{}{-}}\NormalTok{ intercept }\SpecialCharTok{+}\NormalTok{ BMI35\_coefficient }\SpecialCharTok{*}\NormalTok{ BMI\_over\_35}

\CommentTok{\# Convert log odds to probability using logistic function}
\NormalTok{probability\_over\_40 }\OtherTok{\textless{}{-}} \FunctionTok{exp}\NormalTok{(log\_odds) }\SpecialCharTok{/}\NormalTok{ (}\DecValTok{1} \SpecialCharTok{+} \FunctionTok{exp}\NormalTok{(log\_odds))}

\CommentTok{\# Print the result}
\NormalTok{probability\_over\_40}
\end{Highlighting}
\end{Shaded}

\begin{verbatim}
## [1] 0.01980339
\end{verbatim}

As you can see, the probability that an individual is older than 40 if
their BMI is \textgreater35 is around 1\% which is very low likelihood.
This may suggest some form of missing data which we believe is not
random as our results show that the higher in age, the lower the
instances of very high BMI. In fact the maximum BMI observed decreases
inversely with age. We further extrapolated the data using this
information by assuming that this is caused due to health complications
that we will attribute to obesity. this conclusion was uses as the basis
for our health scoring system

\hypertarget{results}{%
\section{Results}\label{results}}

Our results showed what variables are predictors of BMI and Obesity
group, we used this info to construct a scoring system based which
assesses obesity risk. Our result also pointed to the fact that Age is
significantly influenced by an individuals Obesity level especially if
they were classified as Obesity Type 2 or 3.

We also gathered enough information to construct a health risk scoring
system which is based upon the variables impact on an individuals
ability to reach an old age.

\hypertarget{scoring-systems}{%
\subsection{Scoring Systems}\label{scoring-systems}}

\hypertarget{obesity-risk-score}{%
\section{Obesity Risk Score}\label{obesity-risk-score}}

Physical Activity Score: Lower scores are assigned for higher amounts of
physical activity, reflecting its role in reducing obesity risk. This
emphasizes the protective effect of physical activity against obesity.

Tech Time Score: Increased screen time, captured as Tech Time, is
associated with higher scores, indicating its negative impact due to
sedentary behavior.

Vegetable Consumption Score: Higher vegetable consumption is rewarded
with lower scores, supporting the role of a plant-rich diet in
maintaining a healthy weight.

Family History Score: A positive family history of overweight or obesity
significantly increases the score, acknowledging the genetic and
environmental influence on obesity risk.

High-Calorie Food Score and Food Between Meals Score: High scores for
high calorie food consumption and frequent eating between meals
highlight their contribution to caloric excess and obesity.

Water Consumption Score: Higher water consumption, indicative of
healthier lifestyle choices, is assigned lower scores.

Monitor Calories Score: Engaging in monitoring calories is seen as a
positive behavior and thus receives a lower score, reflecting its
importance in weight management.

Transportation Use Score: Preference for active modes of transportation
like walking or biking scores lower, aligning with the promotion of
physical activity.

BMI Score: Directly incorporates the WHO classification of BMI into the
scoring, with higher categories of BMI receiving higher scores to
directly account for current weight status in the obesity risk
assessment.

Obesity Risk Score: The aggregate score, with specific weights assigned
to each factor, integrates these diverse elements into a comprehensive
measure of obesity risk. The weights reflect the perceived impact of
each factor on obesity risk, with significant factors like physical
activity, family history, and BMI itself given slightly more influence

\begin{Shaded}
\begin{Highlighting}[]
\FunctionTok{head}\NormalTok{(data6)}
\end{Highlighting}
\end{Shaded}

\begin{verbatim}
##   Gender Age Height Weight Family_History_w_Overweight HiCal_Food_Consump
## 1 Female  21   1.62   64.0                         yes                 no
## 2 Female  21   1.52   56.0                         yes                 no
## 3   Male  23   1.80   77.0                         yes                 no
## 4   Male  27   1.80   87.0                          no                 no
## 5   Male  22   1.78   89.8                          no                 no
## 6   Male  29   1.62   53.0                          no                yes
##   Veggie_Consump Main_Meal_Consump Food_bw_Meals Does_Smoke Water_Consump
## 1              2                 3     Sometimes         no             2
## 2              3                 3     Sometimes        yes             3
## 3              2                 3     Sometimes         no             2
## 4              3                 3     Sometimes         no             2
## 5              2                 1     Sometimes         no             2
## 6              2                 3     Sometimes         no             2
##   Monitor_Calories Physical_Activ_Amt Tech_Time Alcohol_Consump
## 1               no                  0         1              no
## 2              yes                  3         0       Sometimes
## 3               no                  2         1      Frequently
## 4               no                  2         0      Frequently
## 5               no                  0         0       Sometimes
## 6               no                  0         0       Sometimes
##      Transportation_Use       Obesity_Level      BMI Obesity_Risk_Score
## 1 Public_Transportation       Normal_Weight 24.38653               2.45
## 2 Public_Transportation       Normal_Weight 24.23823               2.01
## 3 Public_Transportation       Normal_Weight 23.76543               2.32
## 4               Walking  Overweight_Level_I 26.85185               2.04
## 5 Public_Transportation Overweight_Level_II 28.34238               2.35
## 6            Automobile       Normal_Weight 20.19509               2.19
\end{verbatim}

\hypertarget{health-risk}{%
\section{Health Risk}\label{health-risk}}

This system is specifically designed to assess health risks associated
with high BMI, incorporating similar variables from the obesity risk
system but adding alcohol consumption into the mix. The rationale behind
each variable remains consistent, with a few adjustments in weighting to
accommodate the new variable:

Alcohol Consumption is introduced as a factor with its score increasing
with higher consumption levels due to the caloric intake and lifestyle
impacts associated with alcohol use. The Health Risk Score similarly
aggregates these scores, with adjustments in weights to balance the
impact of each factor, including alcohol consumption. A higher
cumulative score in this system also indicates a higher health risk
related to high BMI.

\begin{Shaded}
\begin{Highlighting}[]
\FunctionTok{head}\NormalTok{(data7)}
\end{Highlighting}
\end{Shaded}

\begin{verbatim}
##   Gender Age Height Weight Family_History_w_Overweight HiCal_Food_Consump
## 1 Female  21   1.62   64.0                         yes                 no
## 2 Female  21   1.52   56.0                         yes                 no
## 3   Male  23   1.80   77.0                         yes                 no
## 4   Male  27   1.80   87.0                          no                 no
## 5   Male  22   1.78   89.8                          no                 no
## 6   Male  29   1.62   53.0                          no                yes
##   Veggie_Consump Main_Meal_Consump Food_bw_Meals Does_Smoke Water_Consump
## 1              2                 3     Sometimes         no             2
## 2              3                 3     Sometimes        yes             3
## 3              2                 3     Sometimes         no             2
## 4              3                 3     Sometimes         no             2
## 5              2                 1     Sometimes         no             2
## 6              2                 3     Sometimes         no             2
##   Monitor_Calories Physical_Activ_Amt Tech_Time Alcohol_Consump
## 1               no                  0         1              no
## 2              yes                  3         0       Sometimes
## 3               no                  2         1      Frequently
## 4               no                  2         0      Frequently
## 5               no                  0         0       Sometimes
## 6               no                  0         0       Sometimes
##      Transportation_Use       Obesity_Level      BMI Health_Risk_Score
## 1 Public_Transportation       Normal_Weight 24.38653              2.41
## 2 Public_Transportation       Normal_Weight 24.23823              2.37
## 3 Public_Transportation       Normal_Weight 23.76543              2.30
## 4               Walking  Overweight_Level_I 26.85185              2.11
## 5 Public_Transportation Overweight_Level_II 28.34238              2.36
## 6            Automobile       Normal_Weight 20.19509              2.21
\end{verbatim}

\includegraphics{Write-up-for-group-project_files/figure-latex/unnamed-chunk-17-1.pdf}
\includegraphics{Write-up-for-group-project_files/figure-latex/unnamed-chunk-17-2.pdf}

\includegraphics{Write-up-for-group-project_files/figure-latex/unnamed-chunk-18-1.pdf}

\includegraphics{Write-up-for-group-project_files/figure-latex/unnamed-chunk-19-1.pdf}
\includegraphics{Write-up-for-group-project_files/figure-latex/unnamed-chunk-19-2.pdf}
\includegraphics{Write-up-for-group-project_files/figure-latex/unnamed-chunk-19-3.pdf}

\includegraphics{Write-up-for-group-project_files/figure-latex/unnamed-chunk-20-1.pdf}
\includegraphics{Write-up-for-group-project_files/figure-latex/unnamed-chunk-20-2.pdf}

\begin{verbatim}
## `stat_bin()` using `bins = 30`. Pick better value with `binwidth`.
\end{verbatim}

\includegraphics{Write-up-for-group-project_files/figure-latex/unnamed-chunk-20-3.pdf}

\hypertarget{conclusions}{%
\section{Conclusions}\label{conclusions}}

While we were able to construct a good scoring system to rank the
variables effect of BMI and overall health we also understand that the
models we used were not fully accurate and thus our scoring systems
contain flaws

First off, our BMI model was only at best 50\% accurate as predicting an
individuals BMI involves much more variables and predictors than we had
access to. In fact, the biggest predictor of weight, which is a key
component in BMI, is calories consumed - burned, data we don't have
access to.

Finally, the age range in out dataset wasn't large or diverse enough to
display the impacts of BMI and the other habits on an individuals life
expectancy. That being said, we understand that our Health rating score
may include some inaccuracy as well.

\hypertarget{references}{%
\section{References}\label{references}}

\begin{enumerate}
\def\labelenumi{\arabic{enumi}.}
\tightlist
\item
  Wickham, H., Çetinkaya-Rundel, M., \& Grolemund, G. (2023). R for Data
  Science: Import, Tidy, transform, visualize, and model data. O'Reilly.
\item
  Non verbal tourists. UCI Machine Learning Repository. (n.d.).
  \url{https://archive.ics.uci.edu/dataset/853/non+verbal+tourists+data}
\item
  R studio help()
\end{enumerate}

\end{document}
